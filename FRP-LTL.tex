\documentclass{article}

% Math Typings
\usepackage{amsmath}
\usepackage{amsfonts}
\usepackage{amssymb}
\usepackage{colonequals}
\usepackage[margin=3cm]{geometry}
\usepackage{bussproofs}
\input{mathdef.tex}
\usepackage[inline]{enumitem}

% for Haskell/Agda code
\usepackage{listings}
\usepackage[usenames,dvipsnames,svgnames,table]{xcolor}

\newcommand{\llens}{[\![}
\newcommand{\rlens}{]\!]}
\newcommand{\Time}{\mathsf{Time}}
\newcommand{\TU}{\mathsf{U}}
\newcommand{\TE}{\mathsf{E}}
\newcommand{\Wit}[1]{$\,\text{\texttt{#1}}\,$}

\newcommand{\U}{\mathcal{U}}
\newcommand{\bN}{$\N$}
\newcommand{\uu}{\text{\_}}
\newcommand{\cons}{\coloncolon}
\newcommand{\GG}{\Gamma}
\newcommand{\Gg}{\gamma}
\newcommand{\GD}{\Delta}
\newcommand{\Gd}{\delta}
\newcommand{\Ga}{\alpha}
\newcommand{\Gb}{\beta}
\newcommand{\Gt}{\tau}
\newcommand{\Gn}{\eta}
\newcommand{\GS}{\Sigma}
\newcommand{\Gs}{\sigma}
\newcommand{\GL}{\Lambda}
\newcommand{\Gl}{\lambda}
\newcommand{\A}[1]{\Pi[#1]\;}
\newcommand{\As}[1]{\Pi\;#1\;}
\newcommand{\E}[1]{\Sigma[#1]\;}
\newcommand{\Es}[1]{\Sigma\;#1\;}
\newcommand{\entails}{\;\;\vdash\;\;}
\newcommand{\lam}[1]{\lambda #1.\;}
\newcommand{\fst}{\mathtt{fst}\,}
\newcommand{\snd}{\mathtt{snd}\,}
\newcommand{\injl}{\mathtt{left}\,}
\newcommand{\injr}{\mathtt{right}\,}
\newcommand{\case}{\mathtt{case}\,}

\makeatletter
\newcommand*\idstyle{%
        \expandafter\id@style\the\lst@token\relax
}
\def\id@style#1#2\relax{%
  \ifcat#1\relax\else
    \ifnum`#1<\uccode`#1%
      \it
    \else
      \ifnum`#1>\uccode`#1%
        \it
      \fi
    \fi
  \fi
}
\makeatother

\lstdefinelanguage{HaskellUlisses} {
	basicstyle=\linespread{1.2}\selectfont\small,
	columns=fullflexible,     %%
	keepspaces=true,          %%
	identifierstyle=\idstyle, %%
	morekeywords={data,class,instance,case,of,where,let,in,do,if,then,else,type,with,newtype,syntax},
	sensitive=true,
	morecomment=[l][\small\it\textbf]{--},
	morecomment=[s][\small\it\textbf]{\{-}{-\}},
	morestring=[b]",
	stringstyle=\texttt,
	showstringspaces=false,
	numberstyle=\tiny,
	numberblanklines=true,
	showspaces=false,
	breaklines=true,
	showtabs=false,
  aboveskip=4pt,
  belowskip=-13pt
}

\lstnewenvironment{code}{\lstset{language=HaskellUlisses,mathescape=true}}{}

%% packages & settings for xeCJK
\usepackage{setspace}
\usepackage{fontspec}
\usepackage{xeCJK}
\setCJKmainfont[AutoFakeBold=4,AutoFakeSlant=.3]{微軟正黑體}
\XeTeXlinebreaklocale "zh"
\XeTeXlinebreakskip = 0pt plus 1pt

\setlength{\parindent}{16pt}
\setlength{\parskip}{5pt}
\linespread{1.3}

\begin{document}
\title{LTL types FRP 閱讀報告}
\author{}
\date{}
\maketitle

\section{簡介}
\section{Dependent Types}
  depenent types
  types as values
  type level functions
  pi type
  sigma type
  predicate

\section{Modelling Linear Temporal Logic (LTL)}
  在 Dependently Typed Language 當中,LTL 命題可以很直接的 encode 在型別
  當中。 LTL 命題直接就是 $\Time$ 上面的 predicate,型別為 $\Time\to\U$。
  與此同時,我們可以直接給 LTL 的語法與語義轉換成程式實作如下:

  \begin{code}
  T, F : $\Time\to\U$
  T t = $\top$
  F t = $\bot$

  $(\cdot\land\cdot), (\cdot\lor\cdot), (\cdot\Rightarrow\cdot)$ : $(\Time\to\U)\to(\Time\to\U)\to\Time\to\U$
  (A $\land$ B) t = A t $\times$ B t
  (A $\lor$ B) t = A t + B t
  (A $\Rightarrow$ B) t = A t $\to$ B t

  $\bigcirc,\Diamond,\Box$ : $(\Time\to\U)\to\Time\to\U$
  ($\bigcirc$ A) s = A ($s+1$)
  ($\Diamond$ A) s = $\exists$ {t} ($s\le t$) $\times$ A 
  ($\Box$ A) s = $\forall$ {t} ($s\le t$) $\to$ A t

  $(\cdot\TU\cdot)$ : $(\Time\to\U)\to(\Time\to\U)\to\Time\to\U$
  (A $\TU$ B) s = $\exists$ {t} ($s\le t$) $\times$ A[s,t) $\times$ B t

  $(\cdot[\cdot,\cdot))$ : $(\Time\to\U)\to\Time\to\Time\to\U$
  A[s,u) = $\forall$ {t} ($s\le t$) $\to$ ($t<u$) $\to$ A t
  \end{code}

  其中在以上程式碼中, $(\cdot\times\cdot)$ 是 non-dependent pair 的 data types
  , $(\cdot+\cdot)$ 是 non-dependent sum 的 data types,兩者由 Curry-Howard
  Correspondence 分別對應到邏輯中的 conjunction 以及 disjunction。

  由於型別對應到命題,我們可以看到 LTL 的 modalities 能被直接表述在型別中。
  函數 $A [s,u)$ 則代表著 predicate $A$ 在時間 $[s,u)$ 的半閉半開區間中恆
  成立。與此類似的,還有 $(\cdot[\cdot,\cdot])$ 以及
  $(\cdot\langle\cdot,\cdot\rangle)$ 等函數(其中 $A\langle s,t\rangle$
  代表存在 $t\in [s,u]$ 使得 $A\,t$ 成立)。

  底下我們一併給出其餘的 modalities。需要注意的是在 constructive logic
  當中,原先二值邏輯所具有的 duality 可能不再成立。對於許多 modalities,我們
  直接給出其 constructive 的版本,因為 constructive 的版本中包含了驗證
  其真的成立的證明。

  \begin{code}
  $(\cdot\underline{\TU}\cdot),(\cdot\rhd\cdot),(\cdot\unrhd\cdot),(\cdot\leadsto\cdot)$ : $(\Time\to\U)\to(\Time\to\U)\to\Time\to\U$
  (A $\underline{\TU}$ B) s = $\exists$ {t} ($s\le t$) $\times$ A [s,t] $\times$ B t
  (A $\rhd$ B) s = $\forall$ {t} ($s\le t$) $\to$ A [s,t) $\to$ B t
  (A $\unrhd$ B) s = $\forall$ {t} ($s\le t$) $\to$ A [s,t] $\to$ B t
  (A $\leadsto$ B) s = $\forall$ {s} ($s\le t$) $\to$ A [s,t] $\to$ B $\langle$ s,t $\rangle$
  \end{code}

  特別的是,$A\Rightarrow B$, $A\unrhd B$ 以及 $A\rhd B$ 三者都是
  \emph{function space}:

  \begin{itemize}
    \item 型別 $A\Rightarrow B$ 的元素都是 \emph{stateless function},其
    其在時間 $t$ 時的輸出值只關聯於時間 $t$ 時的輸入值。

    \item 型別 $A\unrhd B$ 的元素是 \emph{casual function},其在時間 $t$
    時的輸出值依賴於輸入值的某一段歷史。

    \item 型別 $A\rhd B$ 的元素是 \emph{decoupled function},其在時間 $t$
    時的輸出值雖然也依賴於輸入的一段歷史,但不可以依賴於時間 $t$ 時的
    輸入值。
  \end{itemize}

  對於以上 LTL 語義的實作,我們有如下的結果:對於任意有 uninterpreted atoms
  $\overrightarrow{A}$ 的 LTL 命題 $F$,我們有

  \[ \text{若 } \vdash p:\forall{\overrightarrow{A}},t.\; F\,t \text{ ,則 } \;F \text{ 是 tautology}\]

  此外, LTL 當中關於「過去」的 modality $\ominus$ (yesterday)、$\boxminus$
  (historically)、$\cdot\mathsf{S}\cdot$ (since) 以及 once (dual to $\Diamond$)
  也是類似的實作出來。

\section{Functional Reactive Programming}
  Functional Reactive Programming (FRP) 是一種響應式編成
  (Reactive programming)
  的型式。相應式編成通常以值與資料的流向為主,相對於函數式或命令式編成中以計算
  或狀態轉移為核心的概念。其中 FRP 的模型是撰寫 Signals (隨時間而變的值)
  上的函數。舉例來說,設 $x,y:\Time\to\R$ 代表使用者滑鼠的座標,則我們可以寫

  \begin{code}
  dist = $\widehat{\mathit{sqrt}}$($\widehat{\mathit{pow}}$(x,2) $\widehat{+}$ $\widehat{\mathit{pow}}$(y,2))
  \end{code}

  來定義隨時間而變的值 \textit{dist}。其中我們把平常的指數函數、根號函數
  point-wise lift 到 Signals ($\Time\to\R$) 上。在此定義之下,我們有對於
  任意 $t\in\Time$,

  \begin{code}
  dist(t) = sqrt(pow(x(t), 2) + pow(y(t), 2))
  \end{code}

  正式的說,一個會產生型別為 $a$ 的值的 \emph{Signal} 為

  \begin{code}
  Signal : $\U\to\U$
  Signal a = $\Time$ $\to$ a
  \end{code}

  所以上例中 \textit{dist} 具有型別 $\text{Signal}\; \R$. 在 Yampa 的設計
  當中,為了避免直接撰寫 Signal 時可能引入的 time-leak 及 space-leak 問題,
  程式設計師只被允許使用給定的 combinators 撰寫所謂的 \emph{signal functions}
  (\emph{signal transformers}),且沒有直接操作 $\text{Signal}\; a$ 的方法:

  \begin{code}
  SF a b = Signal a $\to$ Signal b
  \end{code}

  FRP 設計中,還有另一項元素:事件。由於在本系統中,時間採取離散的設計
  (有些 FRP 系統中時間的模型是連續的),因此事件與 Signal 是結合在一起的。
  例如具有 $\text{Signal}\;(\text{Maybe}\;a)$ 型別的函式即為事件。舉例來說,
  以下的 Signal \textit{mouseButton} 可以定義出相關的事件
  \textit{mouseClick}。

  \begin{code}
  mouseButton(t) = $\displaystyle \left\{ \begin{array}{cl} \mathit{down} & \text{if\; } t\in [2,5) \\ \mathit{up} & \text{otherwise} \end{array} \right.$

  mouseClick(t) = $\displaystyle \left\{ \begin{array}{cl} \mathit{just\; clicked} & \text{if\; } t=5 \\ \mathit{nothing} & \text{otherwise} \end{array} \right.$
  \end{code}


  \subsection{FRP Combinators}

  在 Yampa 的設計中,

  \subsection{Well-behaved Reactive Functions}
\section{LTL types FRP}
  在前一節當中,我們概略的介紹了 FRP 的想法以及系統架構方法。本節我們將
  介紹核心的內容,即 LTL 與 FRP 程式之間如何透過 Curry-Howard Correspondence
  連繫起來。我們可以看到 FRP 基本上是建構以資料流向和事件發生為主的系統,
  這部份很自然的可以跟 LTL 當中 $\Diamond$、$\Box$ 等語義連結起來,而
  我們將看到 LTL 也將可以很好的限制 FRP Signal Function 的行為,讓奇怪的
  程式無法 type check。

  在非 dependently typed language 當中, FRP 程式的型別只有不動的 Signal、
  SF 等,不過在 dependently typed langauge 當中,很自然的型別也可以被 $\Time$
  所 parameterize。我們稱此型別為 \emph{Reactive Type}。注意到第三節中
  LTL 的構造($(\cdot\land\cdot), (\cdot\lor\cdot), (\cdot\Rightarrow\cdot),\bigcirc,\Box,\Diamond,(\cdot\TU\cdot)$)都是 $\text{RSet}$ 上的函數。
  在這樣的設計之下,Curry-Howard Correspondence 提供給我們的是:「
  \emph{Propositinos with one time parameter (temporal propositions) as reactive types}」[1]。

  \begin{code}
  RSet : $\U_1$
  RSet = $\Time\to\U$

  $\langle\cdot\land\cdot\rangle$ : $\U$ $\to$ RSet
  $\langle$ A $\rangle$ = $\lambda t.$ A

  $\llens\cdot\rlens$ : RSet $\to$ $\U$
  $\llens$ A $\rlens$ = $\forall$ {s} A s
  \end{code}

  Reactive types 的一個例子是包含所有過去的時間的型別
  $\text{Past} : \text{RSet}$,定義為 $\text{Past}\; s = \exists\{t\} (t\le s)$
  。此外我們提供了兩個輔助函數。
  \begin{enumerate*}
    \item $\langle\cdot\rangle$:對任意的 $A:\U$,$\langle A\rangle$ 是個
    constant reactive type,在任意時間點的型別都是 $A$。
    \item $\llens\cdot\rlens$:用來將 Reactive types 變回 types,作法是把
    time 參數給 quantify 起來。對於 $A:\text{RSet}$,我們有 $\Gs\{s\}:A\,t$
    對任意的 $\Gs:\llens A\rlens$。
  \end{enumerate*}

  顯然我們有 $\text{Signal}\,A\approx \llens\langle A\rangle\rlens$。
  此外,我們也可以將 type constructors pointwisely lift 到 Reactive type
  上面,則前述的 \textit{mouseButton} 以及 \textit{mouseClick} 的型別可以改寫如下:

  \begin{code}
  $\TE$ : RSet $\to$ RSet
  $\TE$ a = $\Gl$t. Maybe (a t)

  mouseButton : $\llens\;\langle$MouseButtonState$\rangle\;\rlens$
  mouseClick   : $\llens\;\TE\langle$MouseEvent$\rangle\;\rlens$
  \end{code}

  \begin{code}
  ($\cdot$before$\cdot$) : $\forall$ {A,s,u,v} A[s,v] $\to$ $u\le v$ $\to$ A[s,u]
  ($\Gs$ before $\Wit{u<=v}$) $\Wit{s<=t}$ $\Wit{t<=u}$ = $\Gs$ $\Wit{s<=t}$ $\Wit{t<=u<=v}$

  $(\cdot\$\cdot)$ : $\forall$ {A,B,s,u} $(A\unrhd B)\to A[s,u] \to B[s,u]$
  (f $\$$ $\Gs$) $\Wit{s<=t}$ $\Wit{t<=u}$ = f $\Wit{s<=t}$ ($\Gs$ before $\Wit{t<=u}$)

  arr : $\forall$ {A,B} $\llens\Box(A\Rightarrow B)\Rightarrow (A\unrhd B)\rlens$
  arr f $\Wit{s<=t}$ $\Gs$ = f $\Wit{s<=t}$ ($\Gs$ $\Wit{s<=t}$ $\Wit{t<=t}$)

  identity : $\forall$ {A} $\llens A\unrhd A\rlens$
  identity = arr ($\Gl x.$ x)

  fst : $\forall$ {A,B} $\llens (A\land B) \unrhd A\rlens$
  fst = arr ($\Gl(x,y).$ x)

  snd : $\forall$ {A,B} $\llens (A\land B) \unrhd B\rlens$
  snd = arr ($\Gl(x,y).$ y)

  $(\cdot>\!\!>\!\!>\cdot)$ : $\forall$ {A,B,C} $\llens(A\unrhd B) \Rightarrow  (B\unrhd C)\Rightarrow (A \unrhd C)\rlens$
  (f $>\!\!>\!\!>$ g) $\Wit{s<=t}$ $\Gs$ = g $\Wit{s<=t}$ (f $\$$ $\Gs$)

  $(\cdot\&\!\&\!\&\cdot)$ : $\forall$ {A,B,C} $\llens(A\unrhd B)\Rightarrow (A\unrhd C) \Rightarrow (A \unrhd B\land C) \rlens$
  (f $\&\!\&\!\&$ g) $\Wit{s<=t}$ $\Gs$ = ( f $\Wit{s<=t}$ $\Gs$ , g $\Wit{s<=t}$ $\Gs$ )

  constant

  localTime

  initially

  decouple

  never

  now

  later

  tag

  edge

  hold

  first

  last

  switch

  rswitch
  \end{code}

\section{Remark: 演算法優化}
\section{其餘性質}
\section{參考資料}
\end{document}
