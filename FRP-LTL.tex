\documentclass{article}

% Math Typings
\usepackage{amsmath}
\usepackage{amsfonts}
\usepackage{amssymb}
\usepackage{colonequals}
\usepackage[margin=3cm]{geometry}
\usepackage{bussproofs}
\input{mathdef.tex}
\usepackage[inline]{enumitem}

% for Haskell/Agda code
\usepackage{listings}
\usepackage[usenames,dvipsnames,svgnames,table]{xcolor}

\newcommand{\llens}{[\![}
\newcommand{\rlens}{]\!]}
\newcommand{\Time}{\mathsf{Time}}
\newcommand{\TU}{\mathsf{U}}
\newcommand{\TE}{\mathsf{E}}
\newcommand{\Wit}[1]{\,\text{\texttt{#1}}\,}

\newcommand{\U}{\mathcal{U}}
\newcommand{\bN}{$\N$}
\newcommand{\uu}{\text{\_}}
\newcommand{\cons}{\coloncolon}
\newcommand{\GG}{\Gamma}
\newcommand{\Gg}{\gamma}
\newcommand{\GD}{\Delta}
\newcommand{\Gd}{\delta}
\newcommand{\Ga}{\alpha}
\newcommand{\Gb}{\beta}
\newcommand{\Gt}{\tau}
\newcommand{\Gn}{\eta}
\newcommand{\GS}{\Sigma}
\newcommand{\Gs}{\sigma}
\newcommand{\GL}{\Lambda}
\newcommand{\Gl}{\lambda}
\newcommand{\A}[1]{\Pi[#1]\;}
\newcommand{\As}[1]{\Pi\;#1\;}
\newcommand{\E}[1]{\Sigma[#1]\;}
\newcommand{\Es}[1]{\Sigma\;#1\;}
\newcommand{\entails}{\;\;\vdash\;\;}
\newcommand{\lam}[1]{\lambda #1.\;}
\newcommand{\fst}{\mathtt{fst}\,}
\newcommand{\snd}{\mathtt{snd}\,}
\newcommand{\injl}{\mathtt{left}\,}
\newcommand{\injr}{\mathtt{right}\,}
\newcommand{\case}{\mathtt{case}\,}

\makeatletter
\newcommand*\idstyle{%
        \expandafter\id@style\the\lst@token\relax
}
\def\id@style#1#2\relax{%
  \ifcat#1\relax\else
    \ifnum`#1<\uccode`#1%
      \it
    \else
      \ifnum`#1>\uccode`#1%
        \it
      \fi
    \fi
  \fi
}
\makeatother

\lstdefinelanguage{HaskellUlisses} {
	basicstyle=\linespread{1.2}\selectfont\small,
	columns=fullflexible,     %%
	keepspaces=true,          %%
	identifierstyle=\idstyle, %%
	morekeywords={data,class,instance,case,of,where,let,in,do,if,then,else,type,with,newtype,syntax},
	sensitive=true,
	morecomment=[l][\small\it\textbf]{--},
	morecomment=[s][\small\it\textbf]{\{-}{-\}},
	morestring=[b]",
	stringstyle=\texttt,
	showstringspaces=false,
	numberstyle=\tiny,
	numberblanklines=true,
	showspaces=false,
	breaklines=true,
	showtabs=false,
  aboveskip=4pt,
  belowskip=-13pt
}

\lstnewenvironment{code}{\lstset{language=HaskellUlisses,mathescape=true}}{}

%% packages & settings for xeCJK
\usepackage{setspace}
\usepackage{fontspec}
\usepackage{xeCJK}
\setCJKmainfont[AutoFakeBold=4,AutoFakeSlant=.3]{微軟正黑體}
\XeTeXlinebreaklocale "zh"
\XeTeXlinebreakskip = 0pt plus 1pt

\setlength{\parindent}{16pt}
\setlength{\parskip}{5pt}
\linespread{1.3}

\begin{document}
\title{LTL types FRP 閱讀報告}
\author{}
\date{}
\maketitle

\section{簡介}
  本篇文章為閱讀 [1] 的心得。 [1] 當中討論 Functional Reactive Programming
  以及 Linear Temporal Logic 之間的 Curry-Howard Correspondence。在此
  correspondence 當中,Linear Temporal Logic 可以很適當的當成 Reactive Type,
  拿來 type Functional Reactive Programs。文章中的程式皆實作在 Agda 上。
  在 Dependently Typed Langauge, Agda 當中,Linear Temporal Logic 可以很自然
  的 encode 成 type,而它的 Modalities 則成為 type-level functions。

\section{Dependent Types}

  Dependently Typed Languages 當中,程式語言允許型別中後面的部份依賴於
  前面的值。同時也允許做 type-level computation,並能把許多邏輯命題
  encode 在 type 當中。詳細的部份可以參見 [3][4]。舉例來說,

  \begin{code}
  predtype : $\N\to\U$
  predtype 0 = $\N$
  predtype (suc n) = $\N\times\N$

  getpred : (n : $\N$) $\to$ predtype n
  getpred 0 = 0
  getpred (suc n) = (suc n , n)
  \end{code}

  以上的程式碼當中 \textit{predtype} 做了 type-level
  computation,根據輸入的值,計算出不同的 \emph{type},而 \textit{getpred}
  是一個 dependent function,輸入的參數型別是 $\N$,且在 type
  當中,該參數的名字是 $n$,我們並以此計算 \textit{getpred} 最終的回傳值型別
  $\mathit{predtype}\,n$。這裡 $\U$ 代表 \emph{Type of Types} (roughly),即
  universes (雖然實際上應該是 hierarchy of universes,不過這裡略去不提)

  \begin{minipage}[t]{0.43\textwidth}
    \begin{prooftree}
    \AxiomC{$\GG,a:A \entails e : B\; a$}
    \UnaryInfC{$\GG \entails \lam{a} e : \A{x:A} B\;x$}
    \end{prooftree}
  \end{minipage}
  \begin{minipage}[t]{0.43\textwidth}
    \begin{prooftree}
    \AxiomC{$\GG \entails f : \A{x:A} B\; x$}
    \AxiomC{$\GG \entails a : A$}
    \BinaryInfC{$\GG \entails f\; a : B\; a$}
    \end{prooftree}
  \end{minipage}

  \begin{minipage}{0.35\textwidth}
    \begin{prooftree}
      \AxiomC{$\GG\entails a:A$}
      \AxiomC{$\GG\entails b:B\;a$}
      \BinaryInfC{$\GG\entails (a,b):\E{x:A} B\;x$}
    \end{prooftree}
  \end{minipage}
  \begin{minipage}{0.28\textwidth}
    \begin{prooftree}
      \AxiomC{$\GG\entails p:\E{x:A} B\;x$}
      \UnaryInfC{$\GG\entails\fst p:A$}
    \end{prooftree}
  \end{minipage}
  \begin{minipage}{0.28\textwidth}
    \begin{prooftree}
      \AxiomC{$\GG\entails p:\E{x:A} B\;x$}
      \UnaryInfC{$\GG\entails\snd p:B\;(\fst p)$}
    \end{prooftree}
  \end{minipage}

  \paragraph{$\Pi$ type} 前面的 $(n:\N)\to T(n)$ 就是 $\Pi$ type,dependent
  product type 。$\Pi$ type 有時候又寫成 $\A{n:\N} T$。在邏輯上,$\Pi$ type
  對應到 $\forall$-quantification,因此有時符號又寫作 $\forall\{n\}\;T$
  注意此例當中我們以大括弧將 $n$ 包起來,代表 $n$ 是
  \emph{implicit parameter},由 compiler 自動 infer。$\Pi$ type 中的元素
  即為 dependent function。

  \paragraph{$\Sigma$ type} $\Sigma$ type 寫成 $\E{x:A} B(x)$,是 dependent
  pair,其中的元素為 $(x,p)$。這當中 $x:A$ 是 witness,而 $p:B(x)$ 則是證明。
  $\Sigma$ type 邏輯上對應到 $\exists$-quantification,因此我們這裡的符號
  又寫作 $\exists\{x\}\;B$。

  \paragraph{$\Time$ 慣例} 本篇報告當中我們假設 $\Time$ 是離散的、是
  well-ordered、其上的 ordering 是 decidable。$\Time$ 上的 ordering
  是一個 data type。$s\le t$ 代表 $s$ 小於等於 $t$ 這個命題(這個
  predicate 其實是 data type),並以 $\Wit{s<=t}$ 代表它的證明。
  此外 $\Wit{s<=s}$、$\Wit{s<=t<=u}$ 分別代表 $(\cdot\le\cdot)$ 的
  reflexivity 以及 transitivity。

\section{Modelling Linear Temporal Logic (LTL)}
  在 Dependently Typed Language 當中,LTL 命題可以很直接的 encode 在型別
  當中。 LTL 命題直接就是 $\Time$ 上面的 predicate,型別為 $\Time\to\U$。
  與此同時,我們可以直接給 LTL 的語法與語義轉換成程式實作如下:

  \begin{code}
  T, F : $\Time\to\U$
  T t = $\top$
  F t = $\bot$

  $(\cdot\land\cdot), (\cdot\lor\cdot), (\cdot\Rightarrow\cdot)$ : $(\Time\to\U)\to(\Time\to\U)\to\Time\to\U$
  (A $\land$ B) t = A t $\times$ B t
  (A $\lor$ B) t = A t + B t
  (A $\Rightarrow$ B) t = A t $\to$ B t

  $\bigcirc,\Diamond,\Box$ : $(\Time\to\U)\to\Time\to\U$
  ($\bigcirc$ A) s = A ($s+1$)
  ($\Diamond$ A) s = $\exists$ {t} ($s\le t$) $\times$ A 
  ($\Box$ A) s = $\forall${t} ($s\le t$) $\to$ A t

  $(\cdot\TU\cdot)$ : $(\Time\to\U)\to(\Time\to\U)\to\Time\to\U$
  (A $\TU$ B) s = $\exists$ {t} ($s\le t$) $\times$ A[s,t) $\times$ B t

  $(\cdot[\cdot,\cdot))$ : $(\Time\to\U)\to\Time\to\Time\to\U$
  A[s,u) = $\forall${t} ($s\le t$) $\to$ ($t<u$) $\to$ A t
  \end{code}

  其中在以上程式碼中, $(\cdot\times\cdot)$ 是 non-dependent pair 的 data types
  , $(\cdot+\cdot)$ 是 non-dependent sum 的 data types,兩者由 Curry-Howard
  Correspondence 分別對應到邏輯中的 conjunction 以及 disjunction。

  由於型別對應到命題,我們可以看到 LTL 的 modalities 能被直接表述在型別中。
  函數 $A [s,u)$ 則代表著 predicate $A$ 在時間 $[s,u)$ 的半閉半開區間中恆
  成立。與此類似的,還有 $(\cdot[\cdot,\cdot])$ 以及
  $(\cdot\langle\cdot,\cdot\rangle)$ 等函數(其中 $A\langle s,t\rangle$
  代表存在 $t\in [s,u]$ 使得 $A\,t$ 成立)。

  底下我們一併給出其餘的 modalities。需要注意的是在 constructive logic
  當中,原先二值邏輯所具有的 duality 可能不再成立。對於許多 modalities,我們
  直接給出其 constructive 的版本,因為 constructive 的版本中包含了驗證
  其真的成立的證明。

  \begin{code}
  $(\cdot\underline{\TU}\cdot),(\cdot\rhd\cdot),(\cdot\unrhd\cdot),(\cdot\leadsto\cdot)$ : $(\Time\to\U)\to(\Time\to\U)\to\Time\to\U$
  (A $\underline{\TU}$ B) s = $\exists$ {t} ($s\le t$) $\times$ A [s,t] $\times$ B t
  (A $\rhd$ B) s = $\forall${t} ($s\le t$) $\to$ A[s,t) $\to$ B t
  (A $\unrhd$ B) s = $\forall${t} ($s\le t$) $\to$ A[s,t] $\to$ B t
  (A $\leadsto$ B) s = $\forall${s} ($s\le t$) $\to$ A[s,t] $\to$ B $\langle$ s,t $\rangle$
  \end{code}

  特別的是,$A\Rightarrow B$, $A\unrhd B$ 以及 $A\rhd B$ 三者都是
  \emph{function space}:

  \begin{itemize}
    \item 型別 $A\Rightarrow B$ 的元素都是 \emph{stateless function},其
    其在時間 $t$ 時的輸出值只關聯於時間 $t$ 時的輸入值。

    \item 型別 $A\unrhd B$ 的元素是 \emph{casual function},其在時間 $t$
    時的輸出值依賴於輸入值的某一段歷史。

    \item 型別 $A\rhd B$ 的元素是 \emph{decoupled function},其在時間 $t$
    時的輸出值雖然也依賴於輸入的一段歷史,但不可以依賴於時間 $t$ 時的
    輸入值。
  \end{itemize}

  對於以上 LTL 語義的實作,我們有如下的結果:對於任意有 uninterpreted atoms
  $\overrightarrow{A}$ 的 LTL 命題 $F$,我們有

  \[ \text{若 } \vdash p:\forall{\overrightarrow{A}},t.\; F\,t \text{ ,則 } \;F \text{ 是 tautology}\]

  此外, LTL 當中關於「過去」的 modality $\ominus$ (yesterday)、$\boxminus$
  (historically)、$\cdot\mathsf{S}\cdot$ (since) 以及 once (dual to $\Diamond$)
  也是類似的實作出來。

\section{Functional Reactive Programming}
  Functional Reactive Programming (FRP) 是一種響應式編成
  (Reactive programming)
  的型式。相應式編成通常以值與資料的流向為主,相對於函數式或命令式編成中以計算
  或狀態轉移為核心的概念。其中 FRP 的模型是撰寫 Signals (隨時間而變的值)
  上的函數。舉例來說,設 $x,y:\Time\to\R$ 代表使用者滑鼠的座標,則我們可以寫

  \begin{code}
  dist = $\widehat{\mathit{sqrt}}$($\widehat{\mathit{pow}}$(x,2) $\widehat{+}$ $\widehat{\mathit{pow}}$(y,2))
  \end{code}

  來定義隨時間而變的值 \textit{dist}。其中我們把平常的指數函數、根號函數
  point-wise lift 到 Signals ($\Time\to\R$) 上。在此定義之下,我們有對於
  任意 $t\in\Time$,

  \begin{code}
  dist(t) = sqrt(pow(x(t), 2) + pow(y(t), 2))
  \end{code}

  正式的說,一個會產生型別為 $a$ 的值的 \emph{Signal} 為

  \begin{code}
  Signal : $\U\to\U$
  Signal a = $\Time$ $\to$ a
  \end{code}

  所以上例中 \textit{dist} 具有型別 $\text{Signal}\; \R$. 在 Yampa [2] 的設計
  當中,為了避免直接撰寫 Signal 時可能引入的 time-leak 及 space-leak 問題,
  程式設計師只被允許使用給定的 combinators 撰寫所謂的 \emph{signal functions}
  (\emph{signal transformers}),且沒有直接操作 $\text{Signal}\; a$ 的方法:

  \begin{code}
  SF a b = Signal a $\to$ Signal b
  \end{code}

  FRP 設計中,還有另一項元素:事件。由於在本系統中,時間採取離散的設計
  (有些 FRP 系統中時間的模型是連續的),因此事件與 Signal 是結合在一起的。
  例如具有 $\text{Signal}\;(\text{Maybe}\;a)$ 型別的函式即為事件。舉例來說,
  以下的 Signal \textit{mouseButton} 可以定義出相關的事件
  \textit{mouseClick}。

  \begin{code}
  mouseButton(t) = $\displaystyle \left\{ \begin{array}{cl} \mathit{down} & \text{if\; } t\in [2,5) \\ \mathit{up} & \text{otherwise} \end{array} \right.$

  mouseClick(t) = $\displaystyle \left\{ \begin{array}{cl} \mathit{just\; clicked} & \text{if\; } t=5 \\ \mathit{nothing} & \text{otherwise} \end{array} \right.$
  \end{code}


  \subsection{FRP Combinators}

  在 Yampa 的設計中,程式設計師不能直接存取到 signal,也不能直接任意撰寫
  signal functions,只能透過 Yampa 預先提供的 primitives 組合出來。一些範例的
  primitives 為:\textit{arr} (把普通函數 lift 成 signal function)、
  ($\cdot>\!\!>\!\!>\cdot$) (合成 signal functions)、\textit{constant}
  產生固定輸出值的 signal functions)、\textit{delay} (延遲一定時間)

  \begin{code}
  arr : $(a\to b)\to \text{SF}\,a\,b$
  ($\cdot>\!\!>\!\!>\cdot$) : SF a b $\to$ SF b c $\to$ SF a c
  constant : $b\to\text{SF}\,a\,b$
  delay : $\Time\to a\to\text{SF}\,a\,a$
  \end{code}

  在此之外,Yampa 針對事件的處理也提供了許多函式。其中兩個函式 \textit{now}
  產生一個立即的事件、\textit{after} 在指定的時間後產生一次給定的事件,而
  \textit{switch}、\textit{rSwitch} 依據事件可以對資料的流向做切換(可以
  類比於火車的轉轍器)。在 $\text{SF}\,a\,(b,\text{Maybe}\,c)$ 沒有發生事件
  時,這個 signal function 的轉換就是 \textit{switch} 的轉換(前面 $\text{SF}\,a\,(b,\_)$ 這一段)。一旦發生了事件(有 $\text{Just}\,c$,函數 $c\to \text{SF}\,a\,b$ 就會被 apply,產生的新的 signal function 為 \textit{switch} 新的結果。\textit{rSwitch} 也類似,不過每次事件都是產生新的 signal function,
  並會被替換上來 [2, Sec 3. Switches]。

  \begin{code}
  now : $b\to \text{SF}\,a\,(\text{Maybe}\,b)$
  after : $\Time\to b\to \text{SF}\,a\,(\text{Maybe}\,b)$
  switch : $\text{SF}\,a\,(b,\text{Maybe}\,c)\to (c\to\text{SF}\,a\,b)\to\text{SF}\,a\,b$
  rSwitch : $\text{SF}\,a\,b\to\text{SF}\,(a,\text{Maybe}\,(\text{SF}\,a\,b))\,b$
  \end{code}

  \subsection{Well-behaved Reactive Functions}

  Yampa 透過提供一組固定的 primitives 來限制 signal functions $\text{SF}\,a\,b$
  函數能有的型式。然而,型別 $\text{SF}\,a\,b$ 本身並不足以刻畫 signal
  functions (transformations) 函數本身應有的行為。$\text{SF}\,a\,b$ 內部的實
  作也有可能是簡單的 $(\Time\to a)\to\Time\to b$ 函數,在這種狀況下,我們
  並不能保證 signal functions 本身不會取用到未來的函數值。

  不過在 dependently typed language 當中,我們可以透過把函數性質與型別
  包在一起的型式,讓滿足該型別的函數全部帶有某種「正確性」的證明。底下我們
  將 $\text{SF}\,a\,b$ 型別定義為所有滿足 $\textsf{Casual}$ predicate 的函數:

  \begin{code}
  SF a b : $\U\to\U\to\U$
  SF a b = $\E{f: \text{Signal}\, a \to \text{Signal}\, b} \mathsf{Casual}\, f$

  $\mathsf{Casual}$ f = $\forall{\sigma,\sigma'}$ $\forall${t} ($\sigma\approx_t\sigma')\to(f\,\sigma\approx_t f\,\sigma')$
  $\sigma\approx_t\sigma'$ = $\forall${s} ($s\le t$) $\to$ $\sigma\,s=\sigma'\,s$
  \end{code}

  $\mathsf{Casual}$ 限制 signal function $f$ 行為的方法在於要求每個
  $\text{SF}\,a\,b$ 中的函數都要提供證明,保證對於任意的 signal $\sigma$,
  $\sigma'$,對於任意的時間點 $t$,只要他們在 $t$ 以前的表現全都一樣,
  那麼 $f$ 作用在 $\sigma,\sigma'$ 上後在時間點 $t$ 以前,得到的 signal
  表現也必須全部一樣。同樣我們可以延伸此定義,不一定限制 signal 在 $t$
  過去的所有時間點都表現一樣,而是加入一個「起始時間」,只要求在 $[s,t]$
  之間表現一樣:

  \begin{code}
  SF$'$ a b : $\U\to\U\to\Time\to\U$
  SF$'$ a b s = $\E{f: \text{Signal}\, a \to \text{Signal}\, b} \mathsf{Casual}_s\, f$

  $\mathsf{Casual}_s$ f = $\forall{\sigma,\sigma'}$ $\forall${t} ($\sigma\approx_{[s,t]}\sigma')\to(f\,\sigma\approx_{[s,t]} f\,\sigma')$
  $\sigma\approx_{[s,t]}\sigma'$ = $\forall${u} ($s\le u$) $\to$ ($u\le t$) $\to$ $\sigma\,s=\sigma'\,s$
  \end{code}
\section{LTL types FRP}
  在前一節當中,我們概略的介紹了 FRP 的想法以及系統架構方法。本節我們將
  介紹核心的內容,即 LTL 與 FRP 程式之間如何透過 Curry-Howard Correspondence
  連繫起來。我們可以看到 FRP 基本上是建構以資料流向和事件發生為主的系統,
  這部份很自然的可以跟 LTL 當中 $\Diamond$、$\Box$ 等語義連結起來,而
  我們將看到 LTL 也將可以很好的限制 FRP Signal Function 的行為,讓奇怪的
  程式無法 type check。

  在非 dependently typed language 當中, FRP 程式的型別只有不動的 Signal、
  SF 等,不過在 dependently typed langauge 當中,很自然的型別也可以被 $\Time$
  所 parameterize。我們稱此型別為 \emph{Reactive Type}。注意到第三節中
  LTL 的構造($(\cdot\land\cdot), (\cdot\lor\cdot), (\cdot\Rightarrow\cdot),\bigcirc,\Box,\Diamond,(\cdot\TU\cdot)$)都是 $\text{RSet}$ 上的函數。
  在這樣的設計之下,Curry-Howard Correspondence 提供給我們的是:「
  \emph{Propositinos with one time parameter (temporal propositions) as reactive types}」[1]。

  \begin{code}
  RSet : $\U_1$
  RSet = $\Time\to\U$

  $\langle\cdot\land\cdot\rangle$ : $\U$ $\to$ RSet
  $\langle$ A $\rangle$ = $\lambda t.$ A

  $\llens\cdot\rlens$ : RSet $\to$ $\U$
  $\llens$ A $\rlens$ = $\forall${s} A s
  \end{code}

  Reactive types 的一個例子是包含所有過去的時間的型別
  $\text{Past} : \text{RSet}$,定義為 $\text{Past}\; s = \exists\{t\} (t\le s)$
  。此外我們提供了兩個輔助函數。
  \begin{enumerate*}
    \item $\langle\cdot\rangle$:對任意的 $A:\U$,$\langle A\rangle$ 是個
    constant reactive type,在任意時間點的型別都是 $A$。
    \item $\llens\cdot\rlens$:用來將 Reactive types 變回 types,作法是把
    time 參數給 quantify 起來。對於 $A:\text{RSet}$,我們有 $\Gs\{s\}:A\,t$
    對任意的 $\Gs:\llens A\rlens$。
  \end{enumerate*}

  顯然我們有 $\text{Signal}\,A\approx \llens\langle A\rangle\rlens$。
  此外,我們也可以將 type constructors pointwisely lift 到 Reactive type
  上面,則前述的 \textit{mouseButton} 以及 \textit{mouseClick} 的型別可以改寫如下:

  \begin{code}
  $\TE$ : RSet $\to$ RSet
  $\TE$ a = $\Gl$t. Maybe (a t)

  mouseButton : $\llens\;\langle$MouseButtonState$\rangle\;\rlens$
  mouseClick   : $\llens\;\TE\langle$MouseEvent$\rangle\;\rlens$
  \end{code}

  在底下我們給一些 FRP combinators 在此實作中的定義以及它怎麼用 LTL type。
  paper 裡將 \textit{arr}, \textit{fst}, \textit{snd}, $(\cdot\&\!\&\!\&\cdot)$
  都以 $(\cdot\unrhd\cdot)$ 給語義,不過這邊我不太確定為何。當然相對於直接以
  $(\cdot\Rightarrow\cdot)$ 給語義,$(\cdot\unrhd\cdot)$ 多了時間的資訊(有
  $\forall\{t\}\,s\le t$ 這個資訊可以用)。

  \begin{minipage}[t]{\textwidth}
    \begin{minipage}[t]{0.46\textwidth}
      \begin{code}
[$\cdot$] : $\forall${A} $\llens A\rlens \to \llens\Box A\rlens$
[ a ] $\Wit{s<=t}$ = a

($\cdot$before$\cdot$) : $\forall${A,s,u,v} A[s,v] $\to$ $u\le v$ $\to$ A[s,u]
($\Gs$ before $\Wit{u<=v}$) $\Wit{s<=t}$ $\Wit{t<=u}$ = $\Gs$ $\Wit{s<=t}$ $\Wit{t<=u<=v}$

$(\cdot\$\cdot)$ : $\forall${A,B,s,u} $(A\unrhd B)\to A[s,u] \to B[s,u]$
(f $\$$ $\Gs$) $\Wit{s<=t}$ $\Wit{t<=u}$ = f $\Wit{s<=t}$ ($\Gs$ before $\Wit{t<=u}$)

arr : $\forall${A,B} $\llens\Box(A\Rightarrow B)\Rightarrow (A\unrhd B)\rlens$
arr f $\Wit{s<=t}$ $\Gs$ = f $\Wit{s<=t}$ ($\Gs$ $\Wit{s<=t}$ $\Wit{t<=t}$)
      \end{code}
    \end{minipage}
    \begin{minipage}[t]{0.54\textwidth}
      \begin{code}
identity : $\forall${A} $\llens A\unrhd A\rlens$
fst : $\forall${A,B} $\llens (A\land B) \unrhd A\rlens$
snd : $\forall${A,B} $\llens (A\land B) \unrhd B\rlens$

identity = arr [$\Gl x.$ x]
fst = arr [$\Gl(x,y).$ x]
snd = arr [$\Gl(x,y).$ y]

$(\cdot>\!\!>\!\!>\cdot)$ : $\forall${A,B,C} $\llens(A\unrhd B) \Rightarrow  (B\unrhd C)\Rightarrow (A \unrhd C)\rlens$
(f $>\!\!>\!\!>$ g) $\Wit{s<=t}$ $\Gs$ = g $\Wit{s<=t}$ (f $\$$ $\Gs$)

$(\cdot\&\!\&\!\&\cdot)$ : $\forall${A,B,C} $\llens(A\unrhd B)\Rightarrow (A\unrhd C) \Rightarrow (A \unrhd B\land C) \rlens$
(f $\&\!\&\!\&$ g) $\Wit{s<=t}$ $\Gs$ = ( f $\Wit{s<=t}$ $\Gs$ , g $\Wit{s<=t}$ $\Gs$ )
      \end{code}
    \end{minipage}
  \end{minipage}

  我們也可以回顧 4.2 節中討論的 Signal function 在 LTL 當中會如何 typed。
  將證明拿進來考慮並展開定義的話,我們能夠得到以下的比較表:

  \begin{enumerate}
    \item $\text{SF}\,A\,B \approx \llens \boxminus \langle A\rangle \Rightarrow \langle B\rangle\rlens$
    \begin{code}
SF A B = $\E{f : \text{Signal}\, A \to \text{Signal}\, B} \mathsf{Casual}\,f$
           = $\E{f : (\Time\to A)\to(\Time\to B)}$
               $\forall${$\Gs,\Gs',t$} ($\forall${s} ($s\le t$) $\to$ $\Gs\,s=\Gs'\,s$) $\to$ ($\forall${s} ($s\le t$) $\to$ $f\,\Gs\,s=f\,\Gs'\,s$)
$\llens\boxminus\langle A\rangle \Rightarrow \langle B\rangle\rlens$ = $\forall${t} $\boxminus\langle A\rangle$ t $\to$ $\langle B\rangle$ t
                    = $\forall${t} ($\forall${s} ($s\le t$) $\to$ A) $\to$ B
    \end{code}

    大致上來說,對於 $\llens \boxminus\langle A\rangle \Rightarrow \langle B\rangle \rlens$ 中的元素,由於 $\forall\{s\}(s\le t)\to A$ 只能參考到比 $t$ 更前面的時間資訊 $s$,因此我們可以證明其滿足 SF $A$ $B$ 的 Casual
    條件。反過來說,對於 SF $A$ $B$ 中的元素我們顯然可以將其轉換成另一邊的
    元素,只要把證明丟掉就好(而原始 $f$ 的良好性質有賴 Cacual 證明規範)。

    \item $\text{SF}'\,A\,B\approx \langle A\rangle \unrhd \langle B\rangle$
    \begin{code}
SF$'$ A B s = $\E{f : \text{Signal}\, A \to \text{Signal}\, B} \mathsf{Casual}_s\,f$
              = $\E{f : (\Time\to A)\to(\Time\to B)}$
                  $\forall${$\Gs,\Gs',t$} ($\forall${u} ($s\le u$) $\to$ ($u\le t$) $\to$ $\Gs\,u=\Gs'\,u$) $\to$
                                ($\forall${u} ($s\le u$) $\to$ ($u\le t$) $\to$ $f\,\Gs\,u=f\,\Gs'\,u$)
($\langle A\rangle \unrhd \langle B\rangle$) s = $\forall${t} $\langle A\rangle$[s,t] $\to$ $\langle B\rangle$ t
                   = $\forall${t} ($\forall${u} ($s\le u$) $\to$ ($u\le t$) $\to$ A) $\to$ B
    \end{code}

    與前一項類似,不過這次 $\mathsf{Casual}_s\,f$ 條件納入起始時間 $s$。比起
    前項可以看過去所有時間,這次只能看其中一段,因此對照的 LTL type 是
    $\langle A\rangle\unrhd\langle B\rangle$

    \item $\text{SF}''\,A\,B\approx\langle A\rangle \rhd\langle B\rangle$
    與第二項相同,只不過把等號拿掉。

  \end{enumerate}

  我們顯然可見以 LTL 刻畫的 type 更為簡潔。

  \paragraph{FRP Primitives} 前面我們介紹了部份 Yampa 的 FRP primitives,
  這裡我們對應地給出部份 LTL-types-FRP 中的 typing 以及實作。從型別當中我們
  可以看見更多有關時序的資訊表露出來,例如原先 Yampa 的 \textit{constant}
  函數型別(透過 $\text{SF}'\,A\,B$ )對應到$B'\to\llens \langle A'\rangle \unrhd \langle B'\rangle \rlens$,但這裡 LTL 的型別可以是 $\llens\Box B\Rightarrow A\rhd B\rlens\,(\equiv \forall\{s\}(\forall\{t\}(s\le t)\to B\,t)\to \forall\{t\}(s\le t)\to A[s,t)\to B\,t)$(其中 $A',B':\U$、$A,B:\text{RSet}$),型別裡明確的表明了 $B:\text{RSet}$ 必須在未來總是成立,且函數是
  decoupled (不依賴於同一個時間的輸入)。

  \textit{decouple} 函數也可以看出輸入一個 $A$ 後輸出的是 $\ominus A$
  (有 delay,這類似 Yampa 的 \textit{delay} 1)。當然當 $A\equiv\langle A'\rangle$ 時 \textit{decouple}
  的型別會簡化為 $A'\rhd A'$。(\textit{decouple} 的型別展開來為
  $\forall\{A,s\}\; A\,(s-1)\to \forall\{t\}(s\le t)\to A[s,t) \to A\,(t-1)$。)

  \begin{code}
  constant : $\forall${A,B} $\llens \Box B\Rightarrow A\rhd B\rlens$
  constant $\Gt$ $\Wit{s<=t}$ $\Gs$ = $\Gt$ $\Wit{s<=t}$

  decouple : $\forall${A} $\llens\ominus A\Rightarrow A\rhd \ominus A\rlens$
  decouple a $\Wit{s<=t}$ $\Gs$ = if $s\le t-1$ then $\Gs$ $\Wit{s<=t-1}$ $\Wit{t-1<=t}$ else a
  \end{code}
  \paragraph{Event Primitives} 同樣的我們也將 FRP 當中 event 的模型延伸到
  dependently typed 的環境下,並寫與 LTL 結合。在程式碼中,一個 event 的
  型別是 $\TE A$。在時間 $s$ 時,$\TE A$ 有兩個值:沒有事件 \textit{nothing}
  或是事件 \textit{just a}(其中 $a:A\,s$)。

  在 \textit{now} 以及 \textit{later} 函式當中,我們可以看到輸入的參數型別分別是 $B$ 以及 $\Diamond B$(展開來
  是 $B\,s$ 以及 $\exists \{t\} (s\le t)\times B\,t$

  \begin{code}
  never : $\forall${A,B} $\llens A\rhd \TE B\rlens$
  never $\Wit{s<=t}$ $\Gs$ = nothing

  now : $\forall${A,B} $\llens B\Rightarrow A\rhd\TE B\rlens$
  now {s} b {t} $\Wit{s<=t}$ $\Gs$ = if $s=t$ then just b else nothing

  -- $\forall\{s\} (\exists \{u\} (s\le u)\times B\,u) \to \forall\{t\}(s\le t)\to A[s,t)\to \TE B\, t$
  later : $\forall${A,B} $\llens \Diamond B\Rightarrow A\rhd\TE B\rlens$
  later ({u}, ($\_$,b)) {t} $\Wit{s<=t}$ $\Gs$ = if $t = u$ then just b else nothing
  \end{code}

  與 event 相伴隨的就是可以改變 FRP 系統資訊傳播結構的 switch 。這邊需要
  用到 $\Time$ 上的遞迴,實作與證明操作相對麻煩,因此我們忽略實作僅提供型別。
  \textit{switch} 與 \textit{rswitch} 的差別在於 \textit{switch} 只考慮
  第一個 event,而 \textit{rswitch} 考慮每一個 event 。此處 LTL 的型別
  同樣告訴了我們進行替換動作的函式 $g : \Box (C \Rightarrow A \unrhd B)$
  被要求必須能在未來任何時間點執行(並產生出替換用的 $A\unrhd B$ ),因為
  \textit{switch} 的動作有可能在未來任何時間點發生。

  \begin{code}
  first : $\forall${A} $\llens \TE A\leadsto \TE A\rlens$
  last : $\forall${A} $\llens \TE A\leadsto \TE A\rlens$
  switch : $\forall${A,B,C} $\llens (A\unrhd (B\land \TE C))\Rightarrow \Box(C\Rightarrow A\unrhd B) \Rightarrow A\unrhd B\rlens$
  rswitch : $\forall${A,B} $\llens (A\unrhd B) \Rightarrow (A\land \TE(A\unrhd B))\unrhd B\rlens$
  \end{code}
  \paragraph{Loops} 將 FRP 中的 fixed-point combinator 以 LTL 描述時稍微
  麻煩一點,我們這裡也只先給出其型別。由於邏輯中要避免矛盾,我們的 fixed-point
  combinator 只能對 decoupled function 做(不像原先 Haskell 中的實作會
  有任意的 $\bot$ 元素)。

  \begin{code}
  -- $\forall${A,s} ($\forall${u} ($s\le u$) $\to$ A[s,u) $\to$ A u) $\to$ $\forall${t} ($s\le t$) $\to$ A t
  fix : $\forall${A} $\llens (A\rhd A) \Rightarrow \Box A\rlens$
  fix {s} f $\Wit{s<=t}$ = f $\Wit{s<=t}$ $\Gs$ where
      $\Gs$ : $\forall${u} A[s,u)
      $\Gs$ $\Wit{s<=t}$ $\Wit{t<u}$ = f $\Wit{s<=t}$ $\Gs$

  ifix : $\forall${A,B} $\llens ((A\land B)\rhd A) \Rightarrow B\rhd A\rlens$
  loop : $\forall${A,B,C} $\llens ((A\land B)\rhd (A\land C)) \Rightarrow B\rhd C\rlens$
  \end{code}

  在以上的 type 中,decoupled function 有 fixed-point 的事實被表述在
  \textit{fix} 之中,其對於任意(最多只能仰賴上一次的輸入)的函數
  $A\rhd A$ 構造出了其 fixed-point $\Box A$,接著 \textit{ifix} 將其擴充為
  $B-$indexed fixed-point combinator,從而實作出 FRP 中的 \textit{loop}
  combinator。

\section{參考資料}
\begin{enumerate}
  \item Alan Jeffrey. \textit{LTL types FRP: linear-time temporal logic propositions as types, proofs as functional reactive programs}. PLPV '12.
  \item \textit{Yampa}. \texttt{https://wiki.haskell.org/Yampa}.
  Originally developed by Yale Haskell Group.
  \item FLOLAC 2014 Dependently Typed Programming \texttt{http://flolac.iis.sinica.edu.tw/flolac14/dtp.html}
  \item FLOLAC 2014 Type Theory and Logic \texttt{http://flolac.iis.sinica.edu.tw/flolac14/logictype.html}
\end{enumerate}

\end{document}
